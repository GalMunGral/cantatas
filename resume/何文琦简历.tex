\documentclass[10pt]{article}
\usepackage{xeCJK}
\usepackage[margin=4em]{geometry}
\usepackage{titlesec}
\usepackage{enumitem}
\usepackage{blindtext}
\usepackage[T1]{fontenc}

\titleformat{\section}{}{\thesection}{0pt}{\vspace{0pt}}[{\titlerule[0.1pt]}]
\setlist[itemize]{leftmargin=2em, topsep=0.5em, itemsep=0pt, label=$\lambda$}
\setlength\parindent{0pt}
\pagestyle{empty}

\begin{document}

{\bf\huge 何文琦} \hfill 136-2602-2079 | hewenqi96@qq.com

\section*{工作经历}
\textbf{前端工程师, 谷露软件 (Gllue Software), 上海} \hfill 2020/11 -- 2021/11
\begin{itemize}
\item 负责多个产品(含桌面web端和微信小程序) 多个主要模块的前端开发. 主要使用 TypeScript 和 React  (Taro) .
\item 负责在小程序和桌面web端接入谷露一帐通用户系统/统一身份认证中心, 部分场景使用OAuth2.0.
\item 参与前端基础建设: 负责开发(小程序用)轻量级表单引擎和图片生成服务, 参与维护公司自研 XML 编辑/渲染引擎.
\item {\small Redux Saga, SWR, ECharts, Ant Design, Linaria, Sass/LESS, Tailwind CSS, Vite/webpack/Rollup/Parcel, GitLab, ArgoCD}
\end{itemize}

\vspace{0.5em}
\textbf{前端工程师, Étude (\texttt{etudereader.com}), 亚特兰大} \hfill 2019/09 -- 2020/03
\begin{itemize}
\item 负责智能 PDF 阅读器 (Electron应用) 的前端开发, 实现了从原始文本解析目录信息、 划词高亮等功能.
\end{itemize}

\vspace{0.5em}
\textbf{前端实习生, Georgia Institute of Technology, 亚特兰大} \hfill 2019/08 -- 2019/12
\begin{itemize}
\item 参与并主导了期中调查问卷系统的自定义表单功能的开发, 集成了学校 CAS 单点登录系统和 Canvas LMS API.
\item 主要使用 MongoDB, ajv, TypeScript, Angular, Express,  Node.js, Puppeteer.
\end{itemize}

\vspace{0.5em}
\textbf{前端实习生, PegasusCRM, 亚特兰大} \hfill 2018/01 -- 2018/04
\begin{itemize}
\item 主要使用 Laravel, jQuery, Sass/LESS, Vue.  开发了基于Puppeteer 的PDF 生成服务 (Node.js).
\end{itemize}


\section*{教育经历}
\textbf{计算机科学理学学士 (BSCS), Georgia Institute of Technology, 亚特兰大} \hfill 2015/08 -- 2019/12
\begin{itemize}
\item  综合 GPA: 3.97/4.0, 专业 GPA: 4.0/4.0. 辅修物理学 (论文: \texttt{https://galmungral.github.io})
\end{itemize} 

\section*{\texttt{\large github.com/GalMunGral}}

\textbf{\texttt{/telescope}:  (C, Python) 基于 SOCKS 协议的分离式代理 }   \hfill 2021/04
\begin{itemize}
\item 个人日常使用的科学上网工具, 采用Shadowsocks式split proxy架构.
\end{itemize}

\vspace{0.5em}
\textbf{\texttt{/hanbun-lang:} (TypeScript) 为前端设计的文言汇编语言和虚拟机} \hfill 2021/05
\begin{itemize}
\item 设计了面向对象的栈式虚拟机和文言形式的汇编. 解释器采用纯函数式 (parser combinator, freer monad).
\end{itemize}

\vspace{0.5em}
\textbf{\texttt{/replay:} (TypeScript/JavaScript)  前端框架和构建工具(无第三方依赖) } \hfill 2020/05 -- 2020/09
\begin{itemize}
\item 实现了基于 JS 语法的 DSL (使用 Babel 插件), 增量 DOM (原地 diff), 非阻塞渲染, hydration, Vue式依赖收集.
\item 实现了Vite 式本地ESM模式 (支持构建缓存和自动刷新) 和webpack式打包模式 (支持代码分割和动态导入).
\end{itemize}

\vspace{0.5em}
\textbf{\texttt{/react-teletype}  (TypeScript) 多标签页共享的 React 渲染器} \hfill 2021/05 
\begin{itemize}
\item  在 shared worker 内通过消息操作主线程 DOM (节省多个标签页重复加载相同代码并初始化的时间).
\end{itemize}

\vspace{0.5em}
\textbf{\texttt{/colosseum}:  (TypeScript) HTML 5 加密流媒体播放器}   \hfill 2021/10
\begin{itemize}
\item 基于 Media Source API、 DASH 协议和 CryptoJS 实现的视频播放器, 可防止用户下载资源.
\end{itemize}


\vspace{0.5em}
\textbf{\texttt{/layout-sandbox:} (TypeScript) WYSIWYG 拖拽式布局编排工具} \hfill 2021/11

\vspace{0.5em}
\textbf{\texttt{/synk}:  (TypeScript) 严格隔离副作用 (禁止异步代码) 的前端框架} \hfill 2021/04

\vspace{0.5em}
\textbf{\texttt{/web-repl:} (JavaScript) 基于 Socket.IO 的网页端 Python REPL (可用作聊天室)}  \hfill 2018/05 


\vspace{1em}
更多代码见: 
\begin{itemize}
\item \texttt{/preludes} \textit{形式语言、网络协议、前端框架相关的探索和概念验证}.
\item \texttt{/etudes} \textit{简单的应用开发, 包括 web全栈、移动端\textup(原生\textup) 和桌面端 \textup(Electron\textup)}.
\item \texttt{/bagatelles} \textit{零碎代码, 包括简单的 webpack 插件 \textup(按需生成原子化CSS、检测无用文件、分析模块间耦合等功能\textup)}.
\end{itemize}

\end{document}

