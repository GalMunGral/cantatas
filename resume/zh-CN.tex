\documentclass[10pt]{article}
\usepackage{xeCJK}
\usepackage[margin=4em]{geometry}
\usepackage{titlesec}
\usepackage{enumitem}
\usepackage{blindtext}
\usepackage[T1]{fontenc}

\titleformat{\section}{}{\thesection}{0pt}{\vspace{0pt}}[{\titlerule[0.1pt]}]
\setlist[itemize]{leftmargin=2em, topsep=0.5em, itemsep=0pt, label=$\lambda$}
\setlength\parindent{0pt}
\pagestyle{empty}

\begin{document}

{\bf\huge 何文琦} \hfill 136-2602-2079 | hewenqi96@gmail.com

\section*{工作经历}
\textbf{前端工程师, 谷露软件 (Gllue Software), Shanghai} \hfill 2020/11 -- 2021/11
\begin{itemize}
\item 负责web前端和微信小程序多个主要模块的开发, 参与页面可视化编辑器的维护. 主要使用 TypeScript 和 React.
\item 参与前端基础建设: 开发了轻量级表单引擎和图片生成服务, 负责谷露一帐通用户系统/身份认证中心的前端部分.
\item {\small 使用了 Redux Saga, SWR, ECharts, Ant Design, Linaria, Sass/LESS, Tailwind, Vite/webpack/Rollup/Parcel, GitLab, ArgoCD.}
\end{itemize}

\vspace{0.5em}
\textbf{前端 (Electron) 工程师, Étude (\texttt{etudereader.com}), Atlanta} \hfill 2019/09 -- 2020/03
\begin{itemize}
\item 参与 PDF 阅读器的开发, 实现了从原始文本解析目录、 划词高亮等功能.
\end{itemize}

\vspace{0.5em}
\textbf{前端开发实习生, Georgia Institute of Technology, Atlanta} \hfill 2019/08 -- 2019/12
\begin{itemize}
\item 参与并主导了期中调查问卷系统自定义表单功能的开发, 集成了学校 CAS 单点登录系统和 Canvas LMS API.
\item 主要使用 MongoDB, ajv, TypeScript, Angular, Express,  Node.js, Puppeteer.
\end{itemize}

\vspace{0.5em}
\textbf{前端实习生, PegasusCRM, Atlanta} \hfill 2018/01 -- 2018/04
\begin{itemize}
\item 主要使用 Laravel PHP, Vue, jQuery, Sass/LESS, Puppeteer.
\end{itemize}


\section*{教育经历}
\textbf{计算机科学理学学士 (BSCS), Georgia Institute of Technology, Atlanta} \hfill 2015/08 -- 2019/12
\begin{itemize}
\item  专业GPA: 4.0/4.0, 综合 GPA: 3.97/4.0 (辅修物理学)
\end{itemize} 

\section*{\texttt{\large HOME=//github.com/GalMunGral; DEMO=//GalMunGral.github.io}}

\textbf{\texttt{/layouter}: (TypeScript) 简易布局系统与底层 UI 框架}   \hfill 2022/04
\begin{itemize}
\item 独立于 DOM/CSSOM 的 UI 系统, 在 canvas 上实现滚动与弹性视图的重排重绘.
\end{itemize}

\vspace{0.5em}
\textbf{\texttt{/react-teletype}:  (TypeScript) 服务端运行的 React 渲染器} \hfill 2021/05 
\begin{itemize}
\item  在后端运行React应用并通过 WebSocket 同步DOM状态.
\end{itemize}

\vspace{0.5em}
\textbf{\texttt{/replay}: (TypeScript, JavaScript)  自研前端框架和构建工具 } \hfill 2020/05 -- 2020/06
\begin{itemize}
\item 实现了原地 diff 算法, 非阻塞渲染, hydration, 依赖收集. 通过Babel 插件实现了基于 JS 语法的 DSL.
\item 实现了ESM模式的 dev server (支持构建缓存和自动刷新) 和webpack式打包 (支持代码分割和动态导入).
\end{itemize}

\vspace{0.5em}
\textbf{\texttt{/telescope}:  (C) 反审查工具, 基于 SOCKS 协议的分离式代理}   \hfill 2021/04

\vspace{0.5em}
\textbf{\texttt{/hanbun-lang}: (PureScript) 基于文言文, 面向堆栈, 面向对象的解释型编程语言}  \hfill 2021/05

\vspace{0.5em}
\textbf{\texttt{/sketchpad}: (JavaScript) 虚拟指令集与简易的高级语言编译器}  \hfill 2022/05

\vspace{0.5em}
\textbf{\texttt{/turing-machine}:  (Haskell, Common Lisp) 代码实现图灵1936年论文中对通用机的描述} \hfill 2021/12

\vspace{2em}
\textbf{更多代码见 \texttt{/bagatelles}}
\begin{itemize}
\item \texttt{/layout-sandbox:} \textit{WYSIWYG 拖拽式布局编排工具}
\item \texttt{/colosseum}:  \textit{HTML5 加密流媒体 (防盗链) 球形视频播放器 (WebGL)}
\item \texttt{/web-repl:} \textit{网页端 Python REPL 与简易聊天室}
\item \texttt{/*-webpack-plugin:} \textit{按需生成原子化CSS, 检测无用文件, 分析模块间耦合}
\item \texttt{/sitbit:} \textit{移动端久坐时间监控}
\item ...
\end{itemize}

\end{document}

